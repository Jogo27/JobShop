\documentclass{rapport}
\usepackage{josty}


\begin{document}

\title{Jobshop : Méthodes approchées}
\subtitle{Compte-rendu du projet de TP «~Optimisation Combinatoire~»}
\author{Joseph Boudou}
\date{14 janvier 2013}
\maketitle

\section{Présentation}

\section{Évaluation des algorithmes}

\section{Codages}

\section{Heuristique gloutonne}

L'heuristique implémentée est de type «~Earliest Due Date~» (EDD).

Pour chaque job et chaque ressource, on maintient la date minimale de début de la prochaine
opération : $sj_j$ pour chaque job $j$ et $sr_k$ pour chaque ressource $k$. Ces dates sont
initialisées à $0$. Après chaque affectation d'une opération $(i,j)$ à une ressource $k$, on met à
jour ces dates :
$$ sj_j \leftarrow sr_k \leftarrow \max (sj_j, sr_k) + p_{(i,j)} $$

Notons $P$ l'ensemble des opérations qui ont déjà été affectées à une ressource et $k_{(i,j)}$
la ressource nécessaire à l'opération $(i,j)$. On commence par sélectionner pour chaque job la
prochaine opération disponible :
$$ \forall j,~d_j = \min_{(i,j) \notin P} i $$
Pour chacune de ces opérations on détermine la date possible de lancement
$$ \forall j,~so_j = \max (sj_j, sr_{k_{(d_j,j)}}) $$
Ensuite on ne garde que les opérations dont la date $so_j$ est minimale
$$ S_1 = \left\{ j ~|~ so_j = \min_{j'} {so_{j'}} \right\} $$
On calcule la durée restante de travail pour chaque job de $S_1$
$$ \forall j \in S_1,~ W_j = \sum_{i \geq d_j}{p_{(i,j)}} $$
Et on ne garde que les jobs dont la durée restante est maximale
$$ S_2 = \left\{ j \in S_1 ~|~ W_j = \max_{j' \in S_1}{W_{j'}} \right\} $$
Enfin, au cas où $S_2$ ne serait pas un singleton, on sélectione le job d'indice maximum
$$ j^* = \max S_2 $$
On affecte alors $(d_{j^*}, j^*)$ à sa ressource $k$ et on met à jour $P$, $sj_{j}$ et $sr_k$.



\section{Voisinages et recherches locales}

\section{Algorithme génétique}

\section{Conclusion}


\end{document}

% vim: tw=100
